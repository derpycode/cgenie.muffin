%

\documentclass[10pt,twoside]{article}
\usepackage[paper=a4paper,portrait=true,margin=2.5cm,ignorehead,footnotesep=1cm]{geometry}
\usepackage{graphicx}
\usepackage{hyperref}
\usepackage{paralist}
\usepackage{caption}
\usepackage{float}
\usepackage{wasysym}

\linespread{1.1}
\setlength{\pltopsep}{2.5pt}
\setlength{\plparsep}{2.5pt}
\setlength{\partopsep}{2.5pt}
\setlength{\parskip}{2.5pt}

\title{cGENIE Quick-start Guide}
\author{Andy Ridgwell}
\date{\today}

\begin{document}

%=================================================================================================================================
%=== BEGIN DOCUMENT ==============================================================================================================
%=================================================================================================================================

\maketitle

%---------------------------------------------------------------------------------------------------------------------------------
%--- Quick-start guide for cGENIE ------------------------------------------------------------------------------------------------
%---------------------------------------------------------------------------------------------------------------------------------

\begin{compactenum}
\item   To get a (read-only) copy of the current (development) branch of \textit{c}GENIE source code:
\\ From your home directory (or elsewhere, but several path variables will have to be edited -- see below), type:
\vspace{-5pt}\begin{verbatim}
svn co https://svn.ggy.bris.ac.uk/subversion/genie/tags/cgenie.muffin-0.4
--username=genie-user cgenie.muffin
\end{verbatim}\vspace{-5pt}
NOTE: All this must be typed continuously on ONE LINE, with a S P A C E before `\texttt{--username}', and before `\texttt{cgenie}'.
You will be asked for a password -- it is \texttt{g3n1e-user}.

\item   If you are installing on your own machine (i.e. not a Bristol (c)genie friendly cluster) you are likely to have to set a couple of environment variables. The compiler name, netCDF library name, and netCDF path, are specified in the file \texttt{user.mak} (\texttt{genie-main} directory). If the \textit{c}genie code tree (cgenie.muffin) and output directory (cgenie\_output) are installed anywhere other than in your account HOME directory, paths specifying this will have to be edited in: \texttt{user.mak} and \texttt{user.sh} (\texttt{genie-main} directory).
                
\item   Change directory to \texttt{cgenie.muffin/genie-main} and type:
\vspace{-5pt}\begin{verbatim}
make testbiogem
\end{verbatim}\vspace{-5pt}
This compiles a carbon cycle enabled configuration of \textit{c}GENIE and runs a short test, comparing the results against those of a pre-run experiment (also downloaded alongside the model source code). It serves to check that you have the software environment correctly configured. If you are unsuccessful here ... double-check the software and directory environment settings in \texttt{user.mak} or \texttt{user.sh}.

\item   At this point, the science modules are currently compiled in a grid and/or number of tracers configuration that is unlikely to be what you want for running experiments. Clean up all the compiled \textit{c}GENIE modules, ready for re-compiling from the source code, by:
\vspace{-5pt}\begin{verbatim}
make cleanall
\end{verbatim}\vspace{-5pt}
  
\end{compactenum}

That is it as far as basic installation goes. Except to read the \textit{c}GENIE \texttt{User\_manual} ;) (Also see: \textit{c}GENIE \texttt{README} and \textit{c}GENIE \texttt{HOWTO} documents.)

%=================================================================================================================================
%=== END DOCUMENT ================================================================================================================
%=================================================================================================================================

\end{document}
